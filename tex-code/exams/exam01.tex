\documentclass{article}
\usepackage[portuguese]{babel}
\usepackage[utf8]{inputenc}

\author{Yane - TSI}
\title{Resoluções - Prova 1}

\begin{document}

\maketitle
\section{Questão I:}

\begin{flushleft}
Classifique as variáveis abaixo. Obs: Dê a classificação específica. Não classifique apenas em quantitativa e qualitativa:
\begin{end}

\\

\begin{flushleft}
\textbf{a)}Tipo de navegador de internet (google chrome, firefox,edge, etc.).
\\

\textbf{Resp.:}
Qualitativa Nominal.
\end{flushleft}
\\ 
\begin{flushleft}
\textbf{b)} Tempo que uma pessoa passa acessando a internet.
\\
\textbf{Resp.:}
Quantitativa Contínua.
\\ 
\end{flushleft}
\\
\begin{flushleft}
\textbf{c)} Categoria do nível de satisfação dos alunos com a rede wifi do IFPB (muito insatisfeito, pouco insatisfeito, pouco satisfeito, muito satisfeito).

\textbf{Resp.:}
Qualitativa Ordinal.
\end{flushleft}
\\
\begin{flushleft}
\textbf{d)} Percentual de alunos aprovados na disciplina de Estatística.

\textbf{Resp.:}
Qualitativa Contínua.
\end{flushleft} 

\textbf{e)} Número de visitantes em um determinado site.

\textbf{Resp.:}
Quantitativa Discreta.
\\ 
\end{flushleft}
\section{Questão II:}
\\

O rol estatístico abaixo refere-se ao número de minutos que 50 usuários gastaram na internet durante a sua mais recente sessão.
\\
\\

\begin{center}

\begin{tabular}{llllllllll}

07 & 07 & 11 & 17 & 17 & 18 & 19 & 20 & 21 & 22 \\
23 & 28 & 29 & 29 & 30 & 30 & 31 & 31 & 33 & 34 \\
36 & 37 & 39 & 39 & 39 & 40 & 41 & 41 & 42 & 44 \\
44 & 46 & 50 & 51 & 53 & 54 & 54 & 56 & 56 & 56 \\
59 & 62 & 67 & 69 & 72 & 73 & 77 & 78 & 80 & 83 \\

\\
\end{tabular}
\end{center}
\begin{flushleft}

\\
\\

\textbf{a)}
Construa uma distribuição de frequências com 7 classes. Obs: arredonde a amplitude das classes para um número inteiro.
\\

\textbf{Resposta:}

\\

\end{flushleft}
\begin{center}
\begin{tabular}{|c|c|c|c|c|}
\hline 
minutos     & x_{i} & f_{i} & f_{r} (\%) & $x_{i} f_{i} \\ 
\hline 
7 $$\vdash$$ 18 & 12,5 & 5 & 10 & 62,5 \\ 
\hline 
18 $$\vdash$$29  & 23,5 & 7 & 14 & 164,5\\ 
\hline 
29 $$\vdash$$40 & 34,5 & 13 & 26 & 448,5 \\ 
\hline 
40 $$\vdash$$ 51 & 45,4 & 8 & 16 & 364,0\\ 
\hline 
 51 $$\vdash$$ 62 & 56,5 & 8 & 16 & 452,0 \\
\hline 
62 $$\vdash$$ 73 & 67,5 & 4 & 8 & 270,0\\ 
\hline 
73 $$\vdash$$ 84 & 78,5& 5& 10 & 392,5\\ 
\hline
\end{tabular}
\end{center}
\begin{flushleft}

\textbf{b)} É verdade que 60\% dos usuários gastaram pelo menos 40 minutos na internet? Justifique.

\textbf{Resposta:}
Falso, pois valores indicam que percentual correto de quanto os usuários gastaram, ou seja, min$\geq$ 40min corresponde a 16\% + 16\% + 8\% + 10\% $\cong$ 50\%. \\ 

\end{flushleft}
\begin{flushleft}

\textbf{c)} É verdade que 30\% dos usuários gastaram menos de 29 minutos na internet. Justifique.

\\
\textbf{Resposta:}
Falso, pois o percentual correto é de 10\%+14\% $\cong$ 24\%.
\\

\end{flushleft}
\section{Questão III:}
...
\section{Questão IV:}
...
\section{Questão V:}
...
\end{document}








